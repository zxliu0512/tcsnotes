\documentclass[aps,pra,onecolumn,notitlepage,superscriptaddress]{revtex4-1}

%\input{myQcircuit}
\usepackage{graphicx,color}% Include figure files
\usepackage{dcolumn}% Align table columns on decimal point
\usepackage{bm}% bold math
\usepackage{amsmath,amssymb,mathrsfs}
\usepackage{url}
\usepackage{hyperref}
\usepackage{framed}
\usepackage{algorithm}
\usepackage{algpseudocode}
\usepackage{wrapfig}
\usepackage{mathtools}
\usepackage{listings}


\newcommand{\N}{\mathbb{N}}
\newcommand{\Z}{\mathbb{Z}}
\newcommand{\R}{\mathbb{R}}
\newcommand{\C}{\mathbb{C}}
\newcommand{\Q}{\mathbb{Q}}


%  Sets
\newcommand{\set}[1]{\mathsf{#1}}
\newcommand{\grp}[1]{\mathsf{#1}}
\newcommand{\spc}[1]{\mathcal{#1}}

% Integrals

\def\d{{\rm d}}

% Linear structures
\newcommand{\Span}{{\mathsf{Span}}}
\newcommand{\Lin}{\mathsf{Lin}}
\newcommand{\rank}{\mathsf{rank}}

\def\>{\rangle}
\def\<{\langle}
\def\kk{\>\!\>}
\def\bb{\<\!\<}
\def\l{\leftarrow}
\def\r{\rightarrow}
\def\u{\underline}
\def\y{\vdash}
\def\ys{\vdash^*}
\newcommand{\st}[1]{\mathbf{#1}}
\newcommand{\bs}[1]{\boldsymbol{#1}}

% Linear maps
\newcommand{\map}[1]{\mathcal{#1}}
\newcommand{\Tr}{\operatorname{Tr}}
\newcommand{\diag}{\mathsf{diag}}

% languages
\newcommand{\reg}{\mathsf{reg}}
\newcommand{\Lan}{\mathsf{L}}


%  Operational notions
\newcommand{\op}[1]{\operatorname{#1}}

\newcommand{\St}{{\mathsf{St}}}
\newcommand{\Eff}{{\mathsf{Eff}}}
\newcommand{\Pur}{{\mathsf{Pur}}}
\newcommand{\Transf}{{\mathsf{Transf}}}
\newcommand{\Chan}{{\mathsf{Chan}}}


%   By Mo
\newcommand{\arccot}{\mathrm{arccot}\,}

%  Miscellanea
\newcommand\myuparrow{\mathord{\uparrow}}
\newcommand\mydownarrow{\mathord{\downarrow}}
\newcommand\h{{\scriptstyle \frac 12}}

% Environments
\newtheorem{theo}{Theorem}
\newtheorem{ax}{Axiom}
\newtheorem{lemma}{Lemma}
\newtheorem{prop}{Proposition}
\newtheorem{cor}{Corollary}
\newtheorem{defi}{Definition}


\newtheorem{rem}{Remark}
\newtheorem{ex}{Exercise}
\newtheorem{proper}{Property}
\newtheorem{exa}{Example}

\def\Proof{{\bf Proof.~}}
\def\qed{$\blacksquare$ \newline}

\begin{document}
	% \preprint{APS/123-QED}
    \title{UNIFORM COMPUTATION}
    \author{}
    \maketitle
    % \tableofcontents
    % \newpage
    
    \section{Loops and infinity}
    \subsection{Turing machines}
    \subsection{NAND-TM}
    NAND-TM = NAND-CIRC + loops + arrays
    \begin{enumerate}
        \item We add a special integer valued variable $i$. All other variables in NAND-TM will be Boolean valued (as in NAND-CIRC).
        \item We add arrays to the language by allowing variable identifiers to have the form Foo[i] with i being the special integer-valued variable mentioned above. Foo is an array of Boolean values, and Foo[i] refers to the value of this array at location equal to the current value of the variable i.
        \item We use the convention that arrays always start with a capital letter, and scalar variables (which are never indexed with i) start with lowercase letters. Hence Foo is an array and bar is a scalar variable.
        \item The input and output X and Y are now considered arrays with values of zeroes and ones.
        \item We add a special MODANDJUMP instruction that takes two boolean variables $a, b$ as input and does the following:
        \begin{itemize}
            \item a=1,b=1
            \item a=0,b=1
            \item a=1,b=0
            \item a=0,b=0
        \end{itemize}
    \end{enumerate}

    \begin{theo}
        Turing machines and NAND-TM programs are equivalent. For every $F: \{0, 1\}^* \to \{0, 1\}^*$, $F$ is computable by a NAND-TM program $P$ if and only if there is a Turing Machine $M$ that computes $F$.
    \end{theo}

    \section{Equivalent models of computation}
    \subsection{RAM machines and NAND-RAM}
    \subsection{Lambda calculus}
    We start with ``basic expressions'' that contain a single variable such as $x$ or $y$ and build more complex expressions using the following two rules:
    \begin{enumerate}
        \item Application: If $e$ and $e'$ are $\lambda$ expressions, then the $\lambda$ expression $(ee')$ corresponds to applying the function described by $e$ to the input $e'$.
        \item Abstraction: If $e$ is an expression and $x$ is a variable, then the $\lambda$ expression $\lambda x . (e)$ corresponds to the function that on any input $z$ returns the expression $e[x \to z]$ replacing all (free) occurrences of $x$
        in $e$.
    \end{enumerate}

    \begin{defi}[$\lambda$ expression]
        A $\lambda$ expression is either a single variable identifier or an expression that is built from other expressions using the application and abstraction operations.  
    \end{defi}

    \begin{defi}
        [Equivalence of $\lambda$ expressions] Two λ expressions are equivalent if they can be made into the same expression by repeated applications of the following rules:
        \begin{enumerate}
            \item Evaluation ($\beta$ reduction): The expression $(\lambda x.exp)exp'$ is equivalent to $exp[x \to exp']$.
            \item Variable renaming ($\alpha$ conversion): The expression $\lambda x.exp$ is equivalent to $\lambda y.exp[x \to y]$.
        \end{enumerate}
    \end{defi}

    Example: DOUBLE
    \begin{equation}
        DOUBLE \ f = \lambda f.(\lambda x.f(fx))
    \end{equation}

    \subsection{The ``ENHANCE'' $\lambda$ Calculus}
    The enhanced $\lambda$ calculus includes the following set of objects and operations:
    \begin{itemize}
        \item Boolean constants and IF function: The enhanced λ calculus has the constants 0 and 1 and the IF function such that for every $cond \in \{0, 1\}$ and $\lambda$ expressions $a$, $b$, IF cond $a$ $b$ outputs $a$ if $cond = 1$ and outputs b if $cond = 0$.
        \item Pairs: We have the function PAIR such that $PAIR \ x \ y$ returns the pair $(x, y)$ that holds $x$ and $y$.
        \item Lists and strings: Using PAIR we can also construct lists. The idea is that $PAIR\ a\ L$ corresponds to the list obtained by adding the element $a$ to the beginning of a list $L$. A string is simply a list of bits.
        \item List operations: MAP, REDUCE, and FILTER.
        \item Recursion: if we have a function $F$ taking two parameters $me$ and $x$, then RECURSE $F$ will be the function $f$ taking one parameter $x$ such that $f(x) = F (f, x)$ for every $x$.
    \end{itemize}

    \subsubsection{Enhanced $\lambda$ expressions}
    








    
\end{document}