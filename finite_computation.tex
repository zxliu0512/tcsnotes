\documentclass[aps,pra,onecolumn,notitlepage,superscriptaddress]{revtex4-1}

%\input{myQcircuit}
\usepackage{graphicx,color}% Include figure files
\usepackage{dcolumn}% Align table columns on decimal point
\usepackage{bm}% bold math
\usepackage{amsmath,amssymb,mathrsfs}
\usepackage{url}
\usepackage{hyperref}
\usepackage{framed}
\usepackage{algorithm}
\usepackage{algpseudocode}
\usepackage{wrapfig}
\usepackage{mathtools}
\usepackage{listings}


\newcommand{\N}{\mathbb{N}}
\newcommand{\Z}{\mathbb{Z}}
\newcommand{\R}{\mathbb{R}}
\newcommand{\C}{\mathbb{C}}
\newcommand{\Q}{\mathbb{Q}}


%  Sets
\newcommand{\set}[1]{\mathsf{#1}}
\newcommand{\grp}[1]{\mathsf{#1}}
\newcommand{\spc}[1]{\mathcal{#1}}

% Integrals

\def\d{{\rm d}}

% Linear structures
\newcommand{\Span}{{\mathsf{Span}}}
\newcommand{\Lin}{\mathsf{Lin}}
\newcommand{\rank}{\mathsf{rank}}

\def\>{\rangle}
\def\<{\langle}
\def\kk{\>\!\>}
\def\bb{\<\!\<}
\def\l{\leftarrow}
\def\r{\rightarrow}
\def\u{\underline}
\def\y{\vdash}
\def\ys{\vdash^*}
\newcommand{\st}[1]{\mathbf{#1}}
\newcommand{\bs}[1]{\boldsymbol{#1}}

% Linear maps
\newcommand{\map}[1]{\mathcal{#1}}
\newcommand{\Tr}{\operatorname{Tr}}
\newcommand{\diag}{\mathsf{diag}}

% languages
\newcommand{\reg}{\mathsf{reg}}
\newcommand{\Lan}{\mathsf{L}}


%  Operational notions
\newcommand{\op}[1]{\operatorname{#1}}

\newcommand{\St}{{\mathsf{St}}}
\newcommand{\Eff}{{\mathsf{Eff}}}
\newcommand{\Pur}{{\mathsf{Pur}}}
\newcommand{\Transf}{{\mathsf{Transf}}}
\newcommand{\Chan}{{\mathsf{Chan}}}


%   By Mo
\newcommand{\arccot}{\mathrm{arccot}\,}

%  Miscellanea
\newcommand\myuparrow{\mathord{\uparrow}}
\newcommand\mydownarrow{\mathord{\downarrow}}
\newcommand\h{{\scriptstyle \frac 12}}

% Environments
\newtheorem{theo}{Theorem}
\newtheorem{ax}{Axiom}
\newtheorem{lemma}{Lemma}
\newtheorem{prop}{Proposition}
\newtheorem{cor}{Corollary}
\newtheorem{defi}{Definition}


\newtheorem{rem}{Remark}
\newtheorem{ex}{Exercise}
\newtheorem{proper}{Property}
\newtheorem{exa}{Example}

\def\Proof{{\bf Proof.~}}
\def\qed{$\blacksquare$ \newline}

\begin{document}
	% \preprint{APS/123-QED}
    \title{Finite Computation}
    \author{}
    \maketitle
    % \tableofcontents
    % \newpage
    
    \section{Defining computation}
    Computers are stupid. We need to describe algorithms precisely.

    Informal definition of an algorithm: An Algorithm is a set of instructions of how to compute an output from an input by following a sequence of ``elementary steps''. An algorithm $A$ computes a function $F$ if for every input $x$, if we follow the instructions of $A$ on the input $x$, we obtain the output $F(x)$.

    \subsection{Boolean circuit}
    \begin{defi}[Boolan circuits]
        Let $n,m,s$ be positive integers with $s \geq m$. A Boolean circuit with $n$ inputs, $m$ outputs, and $s$ gates, is a labeled directed acyclic graph (DAG) $G = (V, E)$ with $s+n$ vertices satisfying the following properties:
        \begin{enumerate}
            \item Exactly $n$ of the vertices have no in-neighbors. These vertices are know as inputs and are labeled with the $n$ labels: $X[0], \cdots, X[n-1]$.
            \item The other $s$ vertices are known as gates. Each gate is labeled with $\land, \lor$ and $\lnot$. Gates labeled with $\land$ or $\lor$ have two in-neighbors. Gates labeled with $\lnot$ have one in-neighbor. We will allow parallel edges (and so for example an AND gate can have both its in-neighbors be the same vertex).
            \item Exactly $m$ of the gates are also labeled with the $m$ labels $Y[0], \cdots, Y[m-1]$ (in addition to their label $\land, \lor, \lnot$). These are known as outputs.
        \end{enumerate}

        Let $f:\{0,1\}^n \rightarrow \{0,1\}^m$. We say that the circuit $C$ computes $f$ if for every $x \in \{0,1\}^n$, $C(x) = f(x)$.
    \end{defi}

    Briefly speaking, the computation of a Boolean circuit is layer by layer.

    \begin{defi} [AON-CIRC Programming language]
        An AON-CIRC program is a string of lines of the form foo = AND(bar,blah) , foo = OR(bar,blah) and foo = NOT(bar) where foo , bar and blah are variable names. Variables of the form $X[i]$ are known as input variables, and variables of the form $Y[j]$ are known as output variables. In every line, the variables on the righthand side of the assignment operators must either be input variables or variables that have already been assigned a value before.
    \end{defi}

    \begin{theo}[Equivalence of circuits and straight-line programs]
        Let $f: \{0,1\}^n \rightarrow \{0,1\}^m$ and $s \geq m$ be some number. Then $f$ is
        computable by a Boolean circuit of $s$ gates if and only if $f$ is computable by an AON-CIRC program of $s$ lines.
    \end{theo}

    \subsection{The NAND function}
    \begin{theo}
        NAND computes AND, OR, NOT. We can compute AND, OR, and NOT by composing only the NAND function.
    \end{theo}
    \begin{align*}
        NOT(a) &= NAND(a,a) \\
        AND(a,b) &= NOT(NAND(a,b)) \\
        OR(a,b) &= NOT(NOT(NOT(a),NOT(b)))
    \end{align*}

    \begin{theo}
        NAND is a universal operation. For every Boolean circuit $C$ of $s$ gates, there exists a NAND circuit $C'$ of at most $3s$ gates that computes the same function as $C$.
    \end{theo}

    Just like we did for Boolean circuits, we can define a programming-language analog of NAND circuits. It is even simpler than the AON-CIRC language since we only have a single operation. 

    \begin{theo}[Equivalence of NAND circuits and straight-line programs]
        Let $f: \{0,1\}^n \rightarrow \{0,1\}^m$ and $s \geq m$ be some number. Then $f$ is
        computable by a NAND circuit of $s$ gates if and only if $f$ is computable by an NAND-CIRC program of $s$ lines.
    \end{theo}


    
\end{document}